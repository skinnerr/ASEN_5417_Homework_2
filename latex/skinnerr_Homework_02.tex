\documentclass[12pt]{article}

\input{../../Latex_Common/skinnerr_latex_preamble_asen5417.tex}

%%
%% DOCUMENT START
%%

\begin{document}

\pagestyle{fancyplain}
\lhead{}
\chead{}
\rhead{}
\lfoot{ASEN 5417: Homework 2}
\cfoot{\thepage}
\rfoot{Ryan Skinner}

\noindent
{\Large Homework 2}
\hfill
{\large Ryan Skinner}
\\[0.5ex]
{\large ASEN 5417: Numerical Methods}
\hfill
{\large Due 2015/09/17}\\
\hrule
\vspace{6pt}

\section{Introduction}

We solve the following problems to better understand numerical techniques for solving ordinary differential equations. As will be described in the methods section, our tools primarily consist of second- and fourth-order Runge-Kutta methods, and Euler stability analysis.

\subsection{Problem 1}

Equations of motion for a rocket's vertical speed $v$ can be written as
\begin{equation}
(m_c + m_p) \frac{dv}{dt} = -(m_c + m_p) g + m_p v_e - \tfrac{1}{2} \rho v \norm{v} A C_D
,
\label{eq:rocket}
\end{equation}
where $z$ is the vertical coordinate, $v = dz/dt$ is the vertical speed, and
\begin{equation*}
\begin{aligned}
m_c &= 51.02 \text{ kg} &&\text{(rocket casing mass)} \\
g &= 9.8 \text{ m/s}^2 &&\text{(gravitational acceleration)} \\
\rho &= 1.23 \text{ kg/m}^3 &&\text{(air density)} \\
A &= 0.1 \text{ m}^2 &&\text{(maximum cross-sectional area)} \\
v_e &= 360 \text{ m/s} &&\text{(exhaust speed)} \\
C_D &= 0.15 &&\text{(drag coefficient)} \\
m_{p0} &= 102.04 \text{ kg} &&\text{(initial propellant mass)}
.
\end{aligned}
\end{equation*}
Furthermore, the instantaneous propellant mass at time $t$ is given by
\begin{equation}
m_p(t) = m_{p0} - \int_0^t \dot{m}_p dt
,
\end{equation}
and the time-varying burn rate is
\begin{equation}
\dot{m}_p =
\frac{m_{p0}}{4}
\cdot
\begin{cases}
t     & 0 \le t \le 1 \\
1     & 1 \le t \le 4 \\
5 - t & 4 \le t \le 5 \\
0     & 5 \le t
.
\end{cases}
\end{equation}

Use a second-order Runge-Kutta (RK2) integrator with $\Delta t = 0.1 \text{ s}$ to plot $z(t)$ and $v(t)$. Use these plots to find the maximum speed, and the height and time at which it is reached; the maximum height, and the time at which it is reached; and the time and velocity when the rocket hits the ground. Check these results with those obtained from \textsc{Matlab}'s \lstinline|ODE45| solver.

This problem is fairly straight-forward. An RK2 scheme is relatively easy to implement, but we will need to be careful when accounting for the time-dependence of $m_p$. Furthermore, though it is not stated in the problem, when $m_p$ reaches zero, the thrust term involving the exhaust speed $v_e$ in \eqref{eq:rocket} needs to ``turn off.''

\subsection{Problem 2}

Consider the stream function for a two-dimensional jet in self-similar form, which can be written as
\begin{equation}
f''(\eta) + f(\eta) f'(\eta) = 0 ,\qquad
f(0) = 0 ,\qquad
f'(0) = 1
.
\end{equation}
The velocities in the jet can be obtained via
\begin{equation}
\frac{U}{U_0} = f'(\eta) ,\qquad
\frac{V}{U_0} = \eta f'(\eta) - \tfrac{1}{2} f(\eta)
.
\end{equation}
Using numerical integration for $0 \le \eta \le 4$, with a step size of $\Delta \eta = 0.05$, plot the quantities $U/U_0$, $V/V_0$, and $f$ as functions of $\eta$. Perform this analysis with both the second- and fourth-order Runge-Kutta methods, and compare their performance. Both solutions may be compared to those obtained from the best-fit expression
\begin{equation}
f(\eta) = \exp(-0.682 \eta^2)
.
\end{equation}

\subsection{Problem 3}

Show that the equation
\begin{equation}
y'' = -\frac{19}{4} y - 10 y' ,\qquad
y(0) = -9 ,\qquad
y'(0) = 0
\end{equation}
is moderately stiff. Use Euler stability analysis to estimate the largest step size $h_\tmax$ for which the Runge-Kutta method will be stable. Then confirm this estimate by computing $y$ using the fourth-order Runge-Kutta method with $h = \{ \tfrac{1}{2} h_\tmax, 2 h_\tmax\}$. Compare these solutions with the analytical solution.

\section{Methodology}

\section{Results}

\section{Discussion}

\section{References}

No external references were used other than the course notes for this assignment.

\section*{Appendix: MATLAB Code}
The following code listings generate all figures presented in this homework assignment.

%\includecode{Homework_1_Driver.m}



%%
%% DOCUMENT END
%%
\end{document}
